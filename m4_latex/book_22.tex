\documentclass{book}
\usepackage[english]{babel} 
\usepackage[utf8x]{inputenc} 
\usepackage[T1]{fontenc}  
\usepackage{color} 
\usepackage{fancyhdr}
\usepackage{setspace}
\usepackage{amsmath} 
\usepackage{graphicx} 
\usepackage[colorinlistoftodos]{todonotes} 
\usepackage[colorlinks=true, allcolors=blue]{hyperref} 
\usepackage[a4paper,top=3cm,bottom=2cm,left=3cm,right=3cm,marginparwidth=0.5cm]{geometry}
\setcounter{chapter}{5} 
\setcounter{equation}{6}
\definecolor{light-gray}{rgb}{0.8,0.8,0.8} 
\begin{document}
\pagestyle{fancy} 
\renewcommand{\headrulewidth}{0pt}
\fancyhf{}
\textbf{5.2 Fractional changes and low-entropy expressions} \hfill{81}\\
\\
\\
\large{of the plausible alternatives measures the gap between our intuition and
reality; the larger the gap, the harder the correct result must work to fill
it, and the harder we must work to remember the correct result.
\\

Such gaps are the subject of statistical mechanics and information theory
[20, 21], which define the gap as the logarithm of the number of plausible
alternatives and call the logarithmic quantity the entropy. The logarithm
does not alter the essential point that expressions differ in the number of
plausible alternatives and that high-entropy expressions [28]—ones with
many plausible alternatives—are hard to remember and understand.
\\

In contrast, a low-entropy expression allows few plausible alternatives,
and elicits, “Yes! How could it be otherwise?!” Much mathematical and
scientific progress consists of finding ways of thinking that turn highentropy
expressions into easy-to-understand, low-entropy expressions.}
\\

\textit{What is a low-entropy expression for the correction to the product xy?}
\\

\large{A multiplicative correction, being dimensionless, automatically has lower
entropy than the additive correction: The set of plausible dimensionless
expressions is much smaller than the full set of plausible expressions.
\\

The multiplicative correction is $(x + \triangle x )(y + \triangle y)/xy$. As written, this
ratio contains gratuitous entropy. It constructs two dimensioned sums $x + \triangle x$ and $y + \triangle y$, multiplies them, and finally divides the product by $xy$. Although the result is dimensionless, it becomes so only in the last step.
A cleaner method is to group related factors by making dimensionless
quantities right away:}
\begin{equation}
\frac{(x+\Delta x)(y+\Delta y)}{xy}=\frac{x+\Delta x}{x}\frac{y+\Delta y}{y}=(1+\frac{\Delta x}{x})(1+\frac{\Delta y}{y}).
\end{equation}
\large{The right side is built only from the fundamental dimensionless quantity 1
and from meaningful dimensionless ratios: $(\triangle x) / x$ is the fractional change
in $x$, and $\triangle y / y$ is the fractional change in $y$.}
\\

\large{The gratuitous entropy came from mixing $x+\triangle x$, $y+\triangle y,$ $x$ and $y$ willy
nilly, and it was removed by regrouping or unmixing. Unmixing is dif-
ficult with physical systems. Try, for example, to remove a drop of food
coloring mixed into a glass of water. The problem is that a glass of
water contains roughly $10^{25}$ molecules. Fortunately, most mathematical
expressions have fewer constituents. We can often regroup and unmix
the mingled pieces and thereby reduce the entropy of the expression.}

\newpage
 \large\textbf{82} \hfill \textit{5 Taking out the big part} \\ 
\vspace{0pt} 
\\
\\
\colorbox{lightgray}{
\begin{minipage}{\textwidth}
 {\large\textbf{Problem 5.9 Rectangle for the correction factor}
 
Draw a rectangle representing the low-entropy correction factor}
 \begin{equation}
 (1+\frac{\Delta x}{x})(1+\frac{\Delta y}{y}).
 \end{equation}
\end{minipage}}
\\
\\
\large{A low-entropy correction factor produces a low-entropy fractional change:
\begin{equation}
\frac{\Delta (xy)}{xy}=(1+\frac{\Delta x}{x})(1+\frac{\Delta y}{y})-1=\frac{\Delta x}{x}+\frac{\Delta y}{y}+\frac{\Delta x}{x}\frac{\Delta y}{y},
\end{equation}
\\
\\
where $\Delta (xy)/xy$ is the fractional change from $xy$ to $(x+\Delta x)(y+\Delta y)$.
The rightmost term is the product of two small fractions, so it is small
compared to the preceding two terms. Without this small, quadratic term,}
\begin{equation}
\frac{\Delta (xy)}{xy}\approx\frac{\Delta x}{x}+\frac{\Delta y}{y}
\end{equation}
\large{Small fractional changes simply add!
\\

This fractional-change rule is far simpler than the corresponding approximate
rule that the absolute change is $x\Delta y+y\Delta x$. Simplicity indicates
low entropy; indeed, the only plausible alternative to the proposed rule
is the possibility that fractional changes multiply. And this conjecture is
not likely: When $\Delta y=0$, it predicts that $\Delta xy=0$ no matter the value
of $\Delta x$ (this prediction is explored also in Problem 5.12).}
\\

\colorbox{lightgray}{
\begin{minipage}{\textwidth}
 {\textbf{Problem 5.10 Thermal expansion}
 
If, due to thermal expansion, a metal sheet expands in each dimension by 4\%,
what happens to its area?}
\\

\textbf{Problem 5.11 Price rise with a discount}
\\

Imagine that inflation, or copyright law, increases the price of a book by 10\%
compared to last year. Fortunately, as a frequent book buyer, you start getting a
store discount of 15\%. What is the net price change that you see?
\end{minipage}
}
\\
\\
{\textbf{5.2.3 Squaring}
\\

\large{In analyzing the engineered and natural worlds, a common operation is
squaring—a special case of multiplication. Squared lengths are areas, and
squared speeds are proportional to the drag on most objects (Section 2.4):}
\begin{equation}
F_{d}\sim \rho \nu^{2}A
 \end{equation}

\newpage
\textbf{5.2 Fractional changes and low-entropy expressions} \hfill{83}\\
\\
\\
\large{where v is the speed of the object, $A$ is its cross-sectional area, and $\rho$ is
the density of the fluid. As a consequence, driving at highway speeds for
a distance d consumes an energy $E=F_{d} d \sim \rho A \nu^{2}d$. Energy consumption
can therefore be reduced by driving more slowly. This possibility became
important to Western countries in the 1970s when oil prices rose rapidly
(see [7] for an analysis). As a result, the United States instituted a highway
speed limit of 55 mph (90 kph).}
\\
\\
\textit{By what fraction does gasoline consumption fall due to driving 55 mph instead
of 65 mph?}
\\
\\
\large{A lower speed limit reduces gasoline consumption by reducing the drag
force $\rho A \nu^{2}$ and by reducing the driving distance $d$: : People measure and
regulate their commuting more by time than by distance. But finding a
new home or job is a slow process. Therefore, analyze first things first—
assume for this initial analysis that the driving distance $d$ stays fixed (then
try Problem 5.14).}
\\
\\
\large{With that assumption, $E$ is proportional to $\nu^{2}$, and}
\begin{equation}
\frac{\Delta E}{E}=2 \times \frac{\Delta \nu}{\nu}
\end{equation}
\\
\\
\large{Going from 65 mph to 55 mph is roughly a 15\% drop in $\nu$ so the energy
consumption drops by roughly 30\%. Highway driving uses a significant
fraction of the oil consumed by motor vehicles, which in the United States
consume a significant fraction of all oil consumed. Thus the 30\% drop
substantially reduced total US oil consumption.}
\\
\\
\colorbox{lightgray}{
\begin{minipage}{\textwidth}
 {\textbf{Problem 5.12 A tempting error}
 
If $A$ and $x$ are related by $A=x^{2}$ , a tempting conjecture is that}
\begin{equation}
\frac{\Delta A}{A} \approx (\frac{\Delta x}{x})^{2}.
\end{equation}
Disprove this conjecture using easy cases (Chapter 2).
\\
\\
\textbf{Problem 5.13 Numerical estimates }

Use fractional changes to estimate $6.3^{3}$. How accurate is the estimate?
\\
\\
\textbf{Problem 5.14 Time limit on commuting }

Assume that driving time, rather than distance, stays fixed as highway driving
speeds fall by 15\%. What is the resulting fractional change in the gasoline consumed
by highway driving?
\end{minipage}
}
\newpage
 \large\textbf{84} \hfill \textit{5 Taking out the big part} \\ 
\vspace{0pt} 
\\
\\
\colorbox{lightgray}{
\begin{minipage}{\textwidth}
 {\textbf{Problem 5.15 Wind power}
 
The power generated by an ideal wind turbine is proportional to $\nu^{3}$ (why?). If
wind speeds increase by a mere 10\%, what is the effect on the generated power?
The quest for fast winds is one reason that wind turbines are placed on cliffs or
hilltops or at sea.}
\end {minipage}
}
\\
\\

\textbf{5.3 Fractional changes with general exponents}
\\

The fractional-change approximations for changes in $x^{2}$ (Section 5.2.3) and
in $x^{3}$ (Problem 5.13) are special cases of the approximation for $x^{n}$
\begin{equation}
\frac{\Delta (x^{n})}{x^{n}} \approx n \times \frac{\Delta x}{x}
\end{equation}
\\
\\
This rule offers a method for mental division (Section 5.3.1), for estimating
square roots (Section 5.3.2), and for judging a common explanation for the
seasons (Section 5.3.3). The rule requires only that the fractional change
be small and that the exponent $n$ not be too large (Section 5.3.4).
\\
\\
\textbf{5.3.1 Rapid mental division}
\\
\\
\large{The special case $n=-1$ provides the method for rapid mental division.
As an example, let’s estimate $1/13$. Rewrite it as $(x+\Delta x)^{-1}$ with $x=10$ and $\Delta x = 3$. The big part is $x^{-1}= 0.1$. Because $\Delta x / x = 30\%$,  the
fractional correction to $x^{-1}$ is roughly −30\%. The result is 0.07.}
\begin{equation}
\frac{1}{13} \approx \frac{1}{10} - 30\%=0.07,
\end{equation}
\\
\\
\large{where the “−30\%” notation, meaning “decrease the previous object by
30\%,” is a useful shorthand for a factor of 1 − 0.3.}
\\
\\
\textit{How accurate is the estimate, and what is the source of the error?}
\\
\\
\large{The estimate is in error by only 9\%. The error arises because the linear
approximation}
\begin{equation}
\frac{(\Delta x^{-1})}{x^{-1}} \approx -1 \times \frac{\Delta x}{x}
\end{equation}
\\
\\
\large{does not include the square (or higher powers) of the fractional change $\Delta x / x$ (Problem 5.17 asks you to find the squared term).}

\end{document}
